\documentclass[a4paper,11pt]{article}
\usepackage[french]{babel}
\usepackage[T1]{fontenc}
\usepackage[utf8]{inputenc}
\usepackage{fancyhdr}
\usepackage{color}
\usepackage{microtype}
\usepackage{graphicx}
\usepackage{booktabs}  % For better tables
\usepackage{enumitem}  % For better lists
\usepackage{hyperref}  % For clickable links in PDF

\pagestyle{fancy}
\setlength{\headheight}{28pt}




\lhead{\includegraphics[width=2cm]{./img/image.png}}
\rhead{\includegraphics[width=2cm]{./img/claude.png}}

\begin{document}

\title{\huge{\textbf{\textcolor{blue}{Lo Poti Donjon}}}}
\author{\Large{Ubissauphte}}
\date{\today}
\maketitle
\newpage
\tableofcontents
\newpage

\section{Introduction}
\subsection{Contexte du projet}
\subsection{Présentation du projet}

\textbf{Lo Poti Donjon} est un projet ambitieux que nous, Ubissauphte, avons lancé le 13 janvier 2025. Initialement, nous envisagions de créer 
un jeu de bataille navale en ligne. Cependant, après plusieurs réunions au sein du groupe, nous avons décidé de revoir notre approche. 

L'idée de créer un jeu plus complexe avec une patte graphique plus importante a vraiment stimulé notre créativité.

Nous avons donc décidé de créer un jeu roguelike, un sous-genre de jeu de rôle où le joueur explore 
un donjon rempli de monstres qu'il doit combattre pour gagner de l'expérience et des trésors. Notre jeu présente certains avantages et défauts :

\begin{center}
\begin{tabular}{p{5cm}|p{5cm}}
\toprule
\textbf{Avantages} & \textbf{Inconvénients} \\
\midrule
Facile d'accès & Potentiellement addictif \\
Permissif & Aspects punitifs \\
Fluidité optimisée & \\
Portage sur diverses machines & \\
Évolution constante du personnage & \\
\bottomrule
\end{tabular}
\end{center}

\subsection{Conception}
\newpage

\section{Organisation du projet}
\textit{Cette partie traite des moyens organisationnels mis en œuvre permettant la réalisation du projet.}

\subsection{Répartition des tâches}

Concernant la répartition des tâches, nous nous sommes appuyés sur les qualités de chacun pour avancer rapidement et efficacement :

\begin{itemize}[leftmargin=*]
  \item \textbf{Victore Fombertasse} a réalisé les animations du jeu ainsi que l'audio et la création de la structure d'un niveau.
  
  \item \textbf{Souleymane Barry} a créé le système d'inventaires, celui de sauvegarde ainsi que le prototype du système de déplacement du joueur.
  
  \item \textbf{Baptiste Carfantan} a conçu les salles générées procéduralement ainsi que la boucle de gameplay pour progresser dans le jeu.
  
  \item \textbf{Barthélemy Deroualt}, chef de projet, s'est chargé de la gestion des systèmes liés aux ennemis, au joueur, aux événements ainsi qu'à la physique du jeu.
\end{itemize}

Nous avons également réalisé des tâches en groupe, que ce soit pour intégrer les codes au programme principal ou pour traiter des problèmes suffisamment complexes nécessitant la participation de plusieurs personnes. Nous avons par exemple développé des systèmes de simplification de code, appelés « managers », permettant à chaque membre de l'équipe d'utiliser les fonctionnalités sans avoir besoin de comprendre entièrement le code des autres.

\subsection{Planning prévisionnel}

\subsection{Outils de travail}

\section{Développement}

\subsection{Génération aléatoire du donjon}

\subsection{Gestion des items}

\subsection{Le moteur physique}

\subsection{Les ennemis}

\subsection{Le cycle de jeu}

\section{Bilan et conclusions}

\section{Annexes}

\subsection{exemple de débogage}
\subsubsection{gestion des pointeurs NULL en fonctions du système d'exploitation}
Nous avons observé une \textbf{erreur de segmentation fault} lors de l'exécution du programme. Fait notable : ce problème n'affectait que certains membres de l'équipe tandis que d'autres pouvaient exécuter l'application normalement.

\begin{itemize}
    \item Les paramètres entrants étaient valides (non-nuls)
    \item Le problème semble provenir soit de :
    \begin{itemize}
        \item La variable \texttt{salleDep}
        \item La structure \texttt{scene}
    \end{itemize}
\end{itemize}

\begin{figure}[h]
    \centering
    \includegraphics[width=0.8\textwidth]{./img/gdbsystème.png}
    \caption{Erreur observée dans GDB}
    \label{fig:gdb_error}
\end{figure}

\begin{figure}[h]
    \centering
    \includegraphics[width=0.8\textwidth]{./img/ExempleMarcheBien.png}
    \caption{Comportement normal pour référence}
    \label{fig:working_example}
\end{figure}

L'inspection de \texttt{salleDep} via GDB révèle :

\begin{figure}[h]
    \centering
    \includegraphics[width=0.8\textwidth]{./img/ErreurAvecPrint.png}
    \caption{Résultat de la commande \texttt{print salleDep}}
    \label{fig:gdb_print}
\end{figure}

\begin{figure}[h]
  \centering
  \includegraphics[width=0.8\textwidth]{./img/Erreur2.png}
  \caption{Résultat de la commande \texttt{print salleDep}}
  \label{fig:gdb_print2}
\end{figure}

Ce qui est marrant avec cette valeur \texttt{0x01} (par exemple) c'est quelle change à chaque fois

\subsection*{Conclusion}
\begin{itemize}
    \item \texttt{salleDep} présente une valeur non-nulle alors que \texttt{sceneController} devrait être vide (\texttt{NULL})
    \item \textbf{Cause racine} : Absence de vérification de nullité avant l'assignation à \texttt{t\_scene}
\end{itemize}

\end{document}