\documentclass[a4paper,11pt]{article}
\usepackage[french]{babel}
\usepackage[T1]{fontenc}
\usepackage[utf8]{inputenc}
\usepackage{fancyhdr}
\usepackage{color}
\usepackage{microtype}
\usepackage{graphicx}
\usepackage{booktabs}  % For better tables
\usepackage{enumitem}  % For better lists
\usepackage{hyperref}  % For clickable links in PDF
\usepackage{amsmath}  % Pour les équations

\pagestyle{fancy}
\setlength{\headheight}{28pt}




\lhead{\includegraphics[width=2cm]{./img/image.png}}
\rhead{\includegraphics[width=2cm]{./img/claude.png}}

\begin{document}

\title{\huge{\textbf{\textcolor{blue}{Lo Poti Donjon}}}}
\author{\Large{Ubissauphte}}
\date{\today}
\maketitle
\newpage
\tableofcontents
\newpage

\section{Introduction}
\subsection{Contexte du projet}
\subsection{Présentation du projet}

\textbf{Lo Poti Donjon} est un projet ambitieux que nous, Ubissauphte, avons lancé le 13 janvier 2025. Initialement, nous envisagions de créer 
un jeu de bataille navale en ligne. Cependant, après plusieurs réunions au sein du groupe, nous avons décidé de revoir notre approche. 

L'idée de créer un jeu plus complexe avec une patte graphique plus importante a vraiment stimulé notre créativité.

Nous avons donc décidé de créer un jeu roguelike, un sous-genre de jeu de rôle où le joueur explore 
un donjon rempli de monstres qu'il doit combattre pour gagner de l'expérience et des trésors. Notre jeu présente certains avantages et défauts :

\begin{center}
\begin{tabular}{p{5cm}|p{5cm}}
\toprule
\textbf{Avantages} & \textbf{Inconvénients} \\
\midrule
Facile d'accès & Potentiellement addictif \\
Permissif & Aspects punitifs \\
Fluidité optimisée & \\
Portage sur diverses machines & \\
Évolution constante du personnage & \\
\bottomrule
\end{tabular}
\end{center}

\subsection{Conception}
\newpage

\section{Organisation du projet}
\textit{Cette partie traite des moyens organisationnels mis en œuvre permettant la réalisation du projet.}

\subsection{Répartition des tâches}

Concernant la répartition des tâches, nous nous sommes appuyés sur les qualités de chacun pour avancer rapidement et efficacement :

\begin{itemize}[leftmargin=*]
  \item \textbf{Victore Fombertasse} a réalisé les animations du jeu ainsi que l'audio et la création de la structure d'un niveau.
  
  \item \textbf{Souleymane Barry} a créé le système d'inventaires, celui de sauvegarde ainsi que le prototype du système de déplacement du joueur.
  
  \item \textbf{Baptiste Carfantan} a conçu les salles générées procéduralement ainsi que la boucle de gameplay pour progresser dans le jeu.
  
  \item \textbf{Barthélemy Deroualt}, chef de projet, s'est chargé de la gestion des systèmes liés aux ennemis, au joueur, aux événements ainsi qu'à la physique du jeu.
\end{itemize}

Nous avons également réalisé des tâches en groupe, que ce soit pour intégrer les codes au programme principal ou pour traiter des problèmes suffisamment complexes nécessitant la participation de plusieurs personnes. Nous avons par exemple développé des systèmes de simplification de code, appelés « managers », permettant à chaque membre de l'équipe d'utiliser les fonctionnalités sans avoir besoin de comprendre entièrement le code des autres.

\subsection{Planning prévisionnel}

\subsection{Outils de travail}

\section{Développement}

\subsection{Génération aléatoire du donjon}

\subsection{Gestion des items}

\subsection{Le moteur physique}

\subsubsection{Pourquoi un moteur physique ?}
\label{sec:moteur_physique}
Un jeu vidéo impliquant des interactions dynamiques entre de nombreux éléments (personnages, ennemis, objets) pose un défi fondamental~: l'imprévisibilité des combinaisons de collisions. Un système simplifié, où chaque interaction serait codée manuellement, deviendrait rapidement ingérable. 

\subsubsection{Gestion des collisions par colliders} 
\label{sec:colliders}
La détection des collisions repose sur une abstraction géométrique des entités. Plutôt que de calculer les intersections \textit{pixel par pixel} -- méthode particulièrement coûteuse en ressources -- chaque élément est associé à un volume primitif. Cette approche répond à trois exigences fondamentales du projet~:

\begin{itemize}
    \item Maintenir un coût de calcul raisonnable.
    \item Garantir des comportements physiques prévisibles.
    \item Faciliter l'extension du système par modularité.
\end{itemize}

\paragraph{Implémentation technique} 
Chaque entité se voit attribuer un composant physique optionnel accompagné d'une forme primitive adaptée à sa nature~:
\begin{itemize}
    \item \textbf{Cercles} pour les éléments arrondis (tonneaux, ennemis, joueur).
    \item \textbf{Rectangles} pour les structures angulaires (caisses, murs).
\end{itemize}
\begin{figure}[ht]
    \centering
    \includegraphics[width=0.75\linewidth]{./img/Capture d'écran 2025-03-29 200818}
    \caption{Visualisation des colliders (en rouge) sur différentes entités}
    \label{fig:colliders_exemple}
\end{figure}

\paragraph{Exemple d'utilisation} 
Le joueur, initialement modélisé par un rectangle de \(16 \times 16\)~px, a été remplacé par un cercle de \(16\)~px de diamètre. Cette modification~:
\begin{itemize}
    \item Procure une sensation de mouvement plus naturelle.
    \item Permet un glissement le long des surfaces.
    \item Génère des rebonds cohérents dans les coins.
\end{itemize}

\paragraph{Exemple d'interaction}  
Lorsqu’un Slime (léger et élastique) est frappé par le joueur, une force orientée selon l’angle de l’attaque lui est appliquée. Projeté contre un Orc (lourd), le moteur calcule automatiquement la réaction : l’Orc subit un léger recul, tandis que le Slime rebondit en changeant de direction. Si le Slime percute ensuite un mur, son élasticité réduit progressivement l’énergie du rebond. Enfin, il peut se cogner contre un autre slime, et lui transmettre encore de l’énergie. Ces interactions en cascade émergent naturellement des lois physiques configurées, sans nécessiter de code spécifique.

\paragraph{Avantages}  
Ce système réduit significativement la complexité du code. Par exemple, une épée qui repousse un ennemi utilise la même logique qu’un projectile ricochant sur un mur, grâce aux paramètres centralisés. Il offre également une extensibilité aisée : ajouter une entité (comme un projectile) se limite à définir ses constantes physiques.

\subsection{Les ennemis}

\subsection{Le cycle de jeu}

\section{Bilan et conclusions}

\section{Annexes}

\subsection{exemple de débogage}
\subsubsection{gestion des pointeurs NULL en fonctions du système d'exploitation}
Nous avons observé une \textbf{erreur de segmentation fault} lors de l'exécution du programme. Fait notable : ce problème n'affectait que certains membres de l'équipe tandis que d'autres pouvaient exécuter l'application normalement.

\begin{itemize}
    \item Les paramètres entrants étaient valides (non-nuls)
    \item Le problème semble provenir soit de :
    \begin{itemize}
        \item La variable \texttt{salleDep}
        \item La structure \texttt{scene}
    \end{itemize}
\end{itemize}

\begin{figure}[h]
    \centering
    \includegraphics[width=0.8\textwidth]{./img/gdbsystème.png}
    \caption{Erreur observée dans GDB}
    \label{fig:gdb_error}
\end{figure}

\begin{figure}[h]
    \centering
    \includegraphics[width=0.8\textwidth]{./img/ExempleMarcheBien.png}
    \caption{Comportement normal pour référence}
    \label{fig:working_example}
\end{figure}

L'inspection de \texttt{salleDep} via GDB révèle :

\begin{figure}[h]
    \centering
    \includegraphics[width=0.8\textwidth]{./img/ErreurAvecPrint.png}
    \caption{Résultat de la commande \texttt{print salleDep}}
    \label{fig:gdb_print}
\end{figure}

\begin{figure}[h]
  \centering
  \includegraphics[width=0.8\textwidth]{./img/Erreur2.png}
  \caption{Résultat de la commande \texttt{print salleDep}}
  \label{fig:gdb_print2}
\end{figure}

Ce qui est marrant avec cette valeur \texttt{0x01} (par exemple) c'est quelle change à chaque fois

\subsection*{Conclusion}
\begin{itemize}
    \item \texttt{salleDep} présente une valeur non-nulle alors que \texttt{sceneController} devrait être vide (\texttt{NULL})
    \item \textbf{Cause racine} : Absence de vérification de nullité avant l'assignation à \texttt{t\_scene}
\end{itemize}

\end{document}